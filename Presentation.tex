\documentclass{beamer}

\mode<presentation> {
\usetheme{Madrid}
}
\usepackage{graphicx}
\usepackage{booktabs} 

\title[YOLO]{Computer Vision with CNN}

\author{Behnia - Heydari}
\institute[AUT]
{
Amirkabir University of Technology \\ 
\medskip
\textit{}
}
\date{May 21, 2019}

\begin{document}

\begin{frame}
\titlepage
\end{frame}

\begin{frame}
\frametitle{Overview}
\tableofcontents
\end{frame}

\section{Motivation}

\begin{frame}
\frametitle{Cameras Everywhere}
\begin{columns}[c]

\column{.4\textwidth}
\textbf{Smartphones}
\begin{itemize}
\item Exploding number of sensors vs. humans

\end{itemize}
\column{.6\textwidth}

\begin{figure}
	\includegraphics[height=180pt]{Pics/samsung.jpg}
\end{figure}
\end{columns}
\end{frame}

\begin{frame}
	\frametitle{Neat Dataset}
	\begin{figure}
		\includegraphics[width=\linewidth]{Pics/image_net.jpeg}
	\end{figure}
\end{frame}

\section{ImageNet Challenge}
\begin{frame}
\frametitle{Role of CNN}
\begin{figure}
	\includegraphics[width=\linewidth]{Pics/imagenet_acc.png}
\end{figure}
\end{frame}

\section{From Cat's Brain to ResNet}
\begin{frame}
\frametitle{Cat's Brain}
\begin{columns}[c]
	\column{.4\textwidth}
	\textbf{Types of cells:}
\begin{itemize}
	\item simple cells
	\item complex cells
	\item hypercomplex cells
\end{itemize}
\column{.6\textwidth}

\begin{figure}
	\includegraphics[width=\linewidth]{Pics/cats.jpg}
\end{figure}
\end{columns}
\end{frame}

\begin{frame}
\frametitle{AlexNet}

	\begin{figure}
		\includegraphics[width=\linewidth]{Pics/alexnet.png}
		\caption{AlexNet: [Krizhevsky, Sutskever, Hinton] 2012}
	\end{figure}

\end{frame}
\begin{frame}
\frametitle{Convolution}

\begin{figure}
	\includegraphics[width=\linewidth]{Pics/cnlayer.png}
	\caption{Convolution Layer}
\end{figure}

\end{frame}

\begin{frame}
\frametitle{Complexity of Features}

\begin{figure}
	\includegraphics[width=\linewidth]{Pics/hlayer.png}
	\caption{complexity of features in each layer}
\end{figure}
\end{frame}

\begin{frame}
\frametitle{Sliding a Filter}
\begin{figure}
	\includegraphics[width=.5\linewidth]{Pics/clook.png}
	\caption{filtering example}
\end{figure}
\end{frame}

\begin{frame}
\frametitle{Sliding a Filter (Cont.)}
\begin{figure}
	\includegraphics[width=.5\linewidth]{Pics/clook2.png}
	\caption{filtering example}
\end{figure}
\end{frame}

\begin{frame}
\frametitle{Sliding a Filter (Cont.)}
\begin{figure}
	\includegraphics[width=.5\linewidth]{Pics/clook3.png}
	\caption{filtering example}
\end{figure}
\end{frame}

\begin{frame}
\frametitle{Sliding a Filter (Cont.)}
 {\color{red}
 What if using a 3*3 filter with stride 3 ?!
	}
\begin{figure}
	\includegraphics[width=.5\linewidth]{Pics/clook4.png}
	\caption{filtering example}
\end{figure}

\end{frame}

\begin{frame}
\frametitle{Padding}
\begin{figure}
	\includegraphics[width=.4\linewidth]{Pics/padding.png}
	\caption{Padding: control output size}
\end{figure}
\end{frame}

\begin{frame}
\frametitle{Summary}
\begin{itemize}
	\item Accept a volume of size N x N 
	\item requires 4 hyper parameters:
	\begin{itemize}
		\item Filter's spatial extent F
		\item Filter's Stride S
		\item Amount of zero padding P
	\end{itemize}
	
     \item Produces a volume of size M X M
     \begin{itemize}
     	\item M = $ \frac{(N - F + 2p)}{S} + 1 $
     	\item Number of parameters : $ (F \times F \times D + b) \times k $
     \end{itemize}
\end{itemize}

\end{frame}
\begin{frame}
\frametitle{Pooling}

\begin{figure}
	\includegraphics[width=\linewidth]{Pics/pooling.png}
	\caption{Max-pooling}
\end{figure}

\end{frame}
\begin{frame}
\frametitle{Fully Connected Layer}

\begin{figure}
	\includegraphics[width=\linewidth]{Pics/cexample.png}
	\caption{Fully Connected Layer}
\end{figure}

\end{frame}
\begin{frame}
\frametitle{AlexNet}

\begin{figure}
	\includegraphics[width=\linewidth]{Pics/alexnetarch.jpg}
	\caption{AlexNet architecture}
\end{figure}

\end{frame}
\begin{frame}
\frametitle{Comparing different Structures}

\begin{figure}
	\includegraphics[width=\linewidth]{Pics/comaprecnn.png}
	\caption{Comparing Structures}
\end{figure}

\end{frame}
\begin{frame}
\frametitle{VGG}
{\color{red} Deeper Networks, Smaller Filters}
\begin{figure}
	\includegraphics[width=.6\linewidth]{Pics/VGG.png}
	\caption{VGG Structures}
\end{figure}

\end{frame}
\begin{frame}
\frametitle{Comparing}

\begin{figure}
	\includegraphics[width=\linewidth]{Pics/compare2.png}
	\caption{Comparing Structures}
\end{figure}

\end{frame}

\begin{frame}
\frametitle{Google Net}
{\color{red} Deeper Networks, with computationally inexpensive}
\begin{figure}
	\includegraphics[width=\linewidth]{Pics/googlenet.png}
     \caption{Googlenet Structures}
\end{figure}
\end{frame}
\begin{frame}
\frametitle{Comparing}

\begin{figure}
	\includegraphics[width=\linewidth]{Pics/com3.png}
	\caption{Comparing Structures}
\end{figure}

\end{frame}
\begin{frame}
\frametitle{ResNet}

\begin{figure}
	\includegraphics[width=.6\linewidth]{Pics/resnet.png}
	\caption{ResNet Structures}
\end{figure}

\end{frame}
\begin{frame}
	
\frametitle{Comparing  Structures}

\begin{figure}
	\includegraphics[width=\linewidth]{Pics/ccomplex.png}
	\caption{Comparing Structures}
\end{figure}

\end{frame}
\begin{frame}
	\frametitle{ResNet}
	
	\begin{figure}
		\includegraphics[width=.6\linewidth]{Pics/resnet.png}
		\caption{ResNet Structures}
	\end{figure}
	
\end{frame}
\begin{frame}
	\frametitle{Class + Location}
	
	\begin{figure}
		\includegraphics[width=\linewidth]{Pics/classloc.PNG}
		
	\end{figure}
	
\end{frame}
\begin{frame}
	\frametitle{Class + Location}
	
	\begin{figure}
		\includegraphics[width=\linewidth]{Pics/loc2.PNG}

	\end{figure}
	
\end{frame}
\begin{frame}
	\frametitle{Example}
	
	\begin{figure}
		\includegraphics[width=\linewidth]{Pics/humanpose.PNG}
		
	\end{figure}
	
\end{frame}
\begin{frame}
	\frametitle{Object detection}
	
	\begin{figure}
		\includegraphics[width=\linewidth]{Pics/detect1.PNG}
		
	\end{figure}
	
\end{frame}
\begin{frame}
	\frametitle{Object detection}
	
	\begin{figure}
		\includegraphics[width=\linewidth]{Pics/objectdetect.PNG}
		
	\end{figure}
	
\end{frame}
\begin{frame}
	\frametitle{Object detection}
	
	\begin{figure}
		\includegraphics[width=\linewidth]{Pics/objectdetect2.PNG}
		
	\end{figure}
	
\end{frame}
\begin{frame}
	\frametitle{RCNN}
	
	\begin{figure}
		\includegraphics[width=\linewidth]{Pics/rcnn.PNG}
		
	\end{figure}
	
\end{frame}
\begin{frame}
	\frametitle{Fast RCNN}
	
	\begin{figure}
		\includegraphics[width=\linewidth]{Pics/fastrcnn.PNG}
		
	\end{figure}
	
\end{frame}
\begin{frame}
	\frametitle{Comparing}
	
	\begin{figure}
		\includegraphics[width=\linewidth]{Pics/crcnn.PNG}
		
	\end{figure}
	
\end{frame}
\begin{frame}
\frametitle{Even Faster}
\begin{figure}
	\includegraphics[width=\linewidth]{Pics/FasterRCNN.png}
\end{figure}
\end{frame}

\begin{frame}
\frametitle{Even Faster}

\begin{figure}
	\includegraphics[width=\linewidth]{Pics/fastercompare.png}
	
\end{figure}

\end{frame}

\begin{frame}
	\frametitle{Yolo / SSD}
	\begin{figure}
		\includegraphics[width= .8\linewidth]{Pics/yolo.PNG}
	\end{figure}
\end{frame}

\begin{frame}
	References:
	\begin{itemize}
		\item Stanford CS231n
	\end{itemize}
\end{frame}

\begin{frame}
\Huge
\centering
Any Question?
\end{frame}

\end{document} 